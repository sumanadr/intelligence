% ****** Start of file aipsamp.tex ******
%
%   This file is part of the AIP files in the AIP distribution for REVTeX 4.
%   Version 4.1 of REVTeX, October 2009
%
%   Copyright (c) 2009 American Institute of Physics.
%
%   See the AIP README file for restrictions and more information.
%
% TeX'ing this file requires that you have AMS-LaTeX 2.0 installed
% as well as the rest of the prerequisites for REVTeX 4.1
%
% It also requires running BibTeX. The commands are as follows:
%
%  1)  latex  aipsamp
%  2)  bibtex aipsamp
%  3)  latex  aipsamp
%  4)  latex  aipsamp
%
% Use this file as a source of example code for your aip document.
% Use the file aiptemplate.tex as a template for your document.
\documentclass[%
 aip,
 jmp,%
 amsmath,amssymb,
%preprint,%
 reprint,%
%author-year,%
%author-numerical,%
]{revtex4-1}

\usepackage{graphicx}% Include figure files
\usepackage{dcolumn}% Align table columns on decimal point
\usepackage{bm}% bold math
%\usepackage[mathlines]{lineno}% Enable numbering of text and display math
%\linenumbers\relax % Commence numbering lines

\begin{document}

\preprint{AIP/123-QED}

\title[Quantum Theory of Love]{Quantum Theory of Love}% Force line breaks with \\
%\thanks{}


\author{Lei Ma}%
 \email{leima@unm.edu.}
\affiliation{ 
Department of Physics and Astronomy, 1919 Lomas Blvd NE, Albuquerque, NM 87131 %\\This line break forced with \textbackslash\textbackslash
}%

\date{\today}% It is always \today, today,
             %  but any date may be explicitly specified

\begin{abstract}
We proposed a hypothesis that emotion can be described by a state vector in Hilbert space and the superposition of emotions is also in the same Hilbert space thus can be interpreted as a kind of emotion. This hypothesis is tested on the description and evolution of love. Then we interpret the results produced by superposition of emotional states. 
\end{abstract}

%\pacs{Valid PACS appear here}% PACS, the Physics and Astronomy
                             % Classification Scheme.
\keywords{quantum, superposition of emotional state, emotion}%Use showkeys class option if keyword
                              %display desired
\maketitle

%\begin{quotation}

%\end{quotation}

\section{First Heading}

In quantum mechanics, the first postulate is usually written as

\begin{quote}
Each physical system is associated with a (topologically) separable complex Hilbert space H with inner product.
\end{quote}

(note: use shankar's form. for later modifications.)


Of course it works so well in physics and we started thinking about apply this kind of formulation to other topics of science.

Human emotions can stack together to give us unique experiences. For example, one can be both happy and sad, be both curious and fear. Similar to quantum mechanics, the first example means superposition of emotions is also a emotion that can be experienced while the second one indicated that we can use direct product of states to describe the complete emotional state.

We now propose a hypothesis of human emotions:

\bf{
Each emotion of humanbeing is associated with a topologically separable complex Hilbert space $\mathbf E$ with inner product.
}




\end{document}
%
% ****** End of file aipsamp.tex ******
